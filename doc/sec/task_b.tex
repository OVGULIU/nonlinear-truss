\section{Task B}
\label{sec:task-b}

The same plots as in section~\ref{sec:task-a} were obtained for the case
when the right truss has a 5\% bigger cross-sectional area,
\(A_{2} = A_{1} \cdot 1.05\), compared to the left truss.
The plots are given in figure~\ref{fig:plots-B} for the left truss.
\begin{figure}[th]
  \centering
  % Force - deflection:
  \begin{subfigure}[t]{\textwidth}
    \begin{tikzpicture}
      \begin{axis}[
        xlabel = {wear depth, [m]},
        ylabel = {count, [-]},
        ]
        \addplot+ table[skip first n=1] {data/force_deflection_elastic_B.dat}; % + = same automatically determined styles, but in addition it uses [options]
      \end{axis}
    \end{tikzpicture}
    \caption{Force - deflection}
  \end{subfigure}

  % Stress - strain:
  \begin{subfigure}[t]{0.48\textwidth}
    \begin{tikzpicture}[baseline,yscale=1.0, xscale=1.0]
      \begin{axis}[
        try min ticks=7,
        minor tick num=1,
        grid=both,
        xlabel = {strain, [-]},
        ylabel = {Kirchhoff stress, [kN/mm\textsuperscript{2}]},
        xmin = -0.4, 
        xmax = 0.3,
        xtick={-0.4, -0.2, 0, 0.2},
        ymin = -30, 
        ymax = 30,
        legend cell align=left,
        legend style={anchor=south east, at={(1,0)}}
        ]
        \addplot+ table[skip first n=1] {data/stress_strain_elastic_B.dat};
        \addlegendentry{elastic}
        \addplot+ table[skip first n=1] {data/stress_strain_plastic_B.dat};
        \addlegendentry{plastic}
      \end{axis}
    \end{tikzpicture}
    \caption{Constitutive behaviour}
  \end{subfigure}
  % Total - plastic strain:
  \begin{subfigure}[t]{0.48\textwidth}
    \begin{tikzpicture}[baseline,yscale=1.0, xscale=1.0]
      \begin{axis}[
        % try min ticks=8,
        minor tick num=1,
        grid=both,
        xlabel = {total strain, [-]},
        ylabel = {plastic strain, [-]},
        xmin = -0.4, 
        xmax = 0.3,
        ymin = -0.27, 
        ymax = 0.15
        ]
        \addplot+ [mark=none] table[skip first n=1] {data/tot_pl_strain_B.dat};
        \addplot+  table[skip first n=1] {data/tot_pl_strain_B.dat}; % Plot again using style of 2nd element in cycle list.
      \end{axis}
    \end{tikzpicture}  
    \caption{Plastic - total strain}
  \end{subfigure}
  \caption{Plots of large deflection elasto-plastic behaviour of the fist truss when \(A_{2} = 1.05 A_{1}\).}
  \label{fig:plots-B}
\end{figure}

In addition, the displacement of the initially upper joint is shown in
figure~\ref{fig:node-displ}.
As could be anticipated, the joint moves in the direction of the truss with
larger cross-sectional area when that truss is stretched.
\begin{figure}[th]
  \centering
  \begin{tikzpicture}[baseline,yscale=1.0, xscale=1.0]
    \begin{axis}[
      % try min ticks=8,
      width = 0.7\textwidth,
      minor tick num=1,
      grid=both,
      xlabel = {horizontal displacement, [mm]},
      ylabel = {vertical displacement, [mm]},
      legend cell align=left,
      legend style={anchor=north east, at={(1,1)}},
      every axis plot/.append style={line width=0.8pt}, % 0.4 - default
      xmin=-12,
      xmax=30,
      ymin=-300,
      ymax=0
      ]
      \addplot+ [mark=none] table[skip first n=1] {data/node_displ_elastic_B.dat};
      \addlegendentry{elastic, \(A_{2} = 1.05 A_{1}\)}
      \addplot+ [mark=none] table[skip first n=1] {data/node_displ_plastic_B.dat};
      \addlegendentry{plastic, \(A_{2} = 1.05 A_{1}\)}
      \addplot+ [mark=none, dashed] table[skip first n=1] {data/node_displ_plastic_B_1percent.dat};
      \addlegendentry{plastic, \(A_{2} = 1.01 A_{1}\)}
      \addplot+ [mark=none, densely dotted] table[skip first n=1] {data/node_displ_plastic_B_10percent.dat};
      \addlegendentry{plastic, \(A_{2} = 1.1 A_{1}\)}
    \end{axis}
  \end{tikzpicture}
  \caption{Displacement of the initially upper node of the first (left) truss.}
  \label{fig:node-displ}
\end{figure}

%%% Local Variables:
%%% mode: latex
%%% TeX-master: "../main"
%%% End:
