\section{Task C}
\label{sec:task-c}

The same plots as in section~\ref{sec:task-b} were obtained for the case
when the truss angle \(\alpha = \operatorname{atan} (1/3)\).
The plots are given in figure~\ref{fig:plots-C} for the left truss.
\begin{figure}[th]
  \centering
  % Force - deflection:
  \begin{subfigure}[t]{\textwidth}
    \begin{tikzpicture}
      \begin{axis}[
        width = 0.95\textwidth,
        height=\axisdefaultheight,
        tick label style={/pgf/number format/fixed},
        try min ticks=6,
        minor tick num=1,
        grid=both,
        xlabel = {\( (Y - y) / L \), [-]},
        ylabel = {\( F / (E A) \), [-]},
        xmin = 0, 
        xmax = 2,
        ymin = -0.15, 
        ymax = 0.2,
        legend cell align=left,
        legend style={anchor=south west, at={(0,0)}}
        ]
        \addplot+ table[skip first n=1] {data/force_deflection_elastic_C.dat}; % + = same automatically determined styles, but in addition it uses [options]
        \addlegendentry{elastic}
        \addplot+ table[skip first n=1] {data/force_deflection_plastic_C.dat};
        \addlegendentry{plastic}
      \end{axis}
    \end{tikzpicture}
    \caption{Force - deflection}
  \end{subfigure}

  % Stress - strain:
  \begin{subfigure}[t]{0.48\textwidth}
    \begin{tikzpicture}[baseline,yscale=1.0, xscale=1.0]
      \begin{axis}[
        try min ticks=5,
        minor tick num=1,
        grid=both,
        xlabel = {strain, [-]},
        ylabel = {Kirchhoff stress, [kN/mm\textsuperscript{2}]},
        % xmin = -0.4, 
        xmax = 0.4,
        % xtick={-0.4, -0.2, 0, 0.2},
        % ymin = -30, 
        % ymax = 30,
        legend cell align=left,
        legend style={anchor=south east, at={(1,0)}}
        ]
        \addplot+ table[skip first n=1] {data/stress_strain_elastic_C.dat};
        \addlegendentry{elastic}
        \addplot+ table[skip first n=1] {data/stress_strain_plastic_C.dat};
        \addlegendentry{plastic}
      \end{axis}
    \end{tikzpicture}
    \caption{Constitutive behaviour}
  \end{subfigure}
  % Total - plastic strain:
  \begin{subfigure}[t]{0.48\textwidth}
    \begin{tikzpicture}[baseline,yscale=1.0, xscale=1.0]
      \begin{axis}[
        tick label style={/pgf/number format/fixed},
        % try min ticks=8,
        minor tick num=1,
        grid=both,
        xlabel = {total strain, [-]},
        ylabel = {plastic strain, [-]},
        % xmin = -0.4, 
        % xmax = 0.2,
        % ymin = -0.1, 
        ymax = 0.01
        ]
        \addplot+ [mark=none] table[skip first n=1] {data/tot_pl_strain_C.dat};
        \addplot+  table[skip first n=1] {data/tot_pl_strain_C.dat}; % Plot again using style of 2nd element in cycle list.
      \end{axis}
    \end{tikzpicture}  
    \caption{Plastic - total strain}
  \end{subfigure}
  \caption{Plots of large deflection elasto-plastic behaviour of the fist truss when \(A_{2} = 1.05 A_{1}\).}
  \label{fig:plots-C}
\end{figure}

In addition, the displacement of the initially upper joint is shown in
figure~\ref{fig:node-displ-C}.
It is evident that for such small angle the solution is sensitive to the
difference in cross-section area between the trusses.
\begin{figure}[th]
  \centering
  \begin{tikzpicture}[baseline,yscale=1.0, xscale=1.0]
    \begin{axis}[
      % try min ticks=8,
      width = 0.7\textwidth,
      minor tick num=1,
      grid=both,
      xlabel = {horizontal displacement, [mm]},
      ylabel = {vertical displacement, [mm]},
      legend cell align=left,
      legend style={anchor=center, at={(0.5,0.5)}},
      every axis plot/.append style={line width=0.8pt}, % 0.4 - default
      % xmin=-12,
      % xmax=30,
      % ymin=-300,
      ymax=0
      ]
      \addplot+ [mark=none] table[skip first n=1] {data/node_displ_elastic_C.dat};
      \addlegendentry{elastic, \(A_{2} = 1.05 A_{1}\)}
      \addplot+ [mark=none] table[skip first n=1] {data/node_displ_plastic_C.dat};
      \addlegendentry{plastic, \(A_{2} = 1.05 A_{1}\)}
      \addplot+ [mark=none, dashed] table[skip first n=1] {data/node_displ_plastic_C_1percent.dat};
      \addlegendentry{plastic, \(A_{2} = 1.01 A_{1}\)}
      \addplot+ [mark=none, densely dotted] table[skip first n=1] {data/node_displ_plastic_C_10percent.dat};
      \addlegendentry{plastic, \(A_{2} = 1.1 A_{1}\)}
    \end{axis}
  \end{tikzpicture}
  \caption{Displacement of the initially upper node of the first (left) truss.}
  \label{fig:node-displ-C}
\end{figure}

Obtaining the convergence for this task was the most problematic.
Even in the case of elastic material response, a situation depicted
in the figure~\ref{fig:no-newton} was occurring, where the local
minimum prevented the Newton-Raphson method from finding the
correct solution.
In those situations, the false position method was used instead.
\begin{figure}[th]
  \centering
  \begin{tikzpicture}[baseline,yscale=1.0, xscale=1.0]
    \begin{axis}[
      width = 0.7\textwidth,
      max space between ticks=50,
      minor tick num=1,
      grid=both,
      xlabel = {displacement, [mm]},
      ylabel = {residual, [N]},
      ]
      \addplot+ table[skip first n=1] {data/residual_vs_du.dat};
    \end{axis}
  \end{tikzpicture}
  \caption{Example of situation, where Newton-Raphson method can fail.}
  \label{fig:no-newton}
\end{figure}


%%% Local Variables:
%%% mode: latex
%%% TeX-master: "../main"
%%% End:
