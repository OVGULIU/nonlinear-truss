\section{Task A}
\label{sec:task-a}

In the first task, the plots from \cite{Bonet2008} were reproduced.
In order to achieve that the presence of the second rod in the given
problem needed to be accounted for.
This was done by normalising the force by the total area of the two rods.
At each displacement increment, the vertical force in the top node (DOF 4)
was computed as a sum of the already acting force plus the force
needed to move the top node by the increment of the displacement,
i.e.
\begin{equation}
  F_{n+1} = F_{n} + K^{\text{global}}_{4,j} \Delta u_{j}
\end{equation}

The plots presented in figure 3.8 a-d of \cite{Bonet2008} were reproduced
and are given in figure~\ref{fig:plots-from-book}.
\begin{figure}[th]
  \centering
  % Force - deflection:
  \begin{subfigure}[t]{\textwidth}
    \begin{tikzpicture}
      \begin{axis}[
        width = 0.95\textwidth,
        height=\axisdefaultheight,
        tick label style={/pgf/number format/fixed},
        try min ticks=6,
        minor tick num=1,
        grid=both,
        xlabel = {\( (Y - y) / L \), [-]},
        ylabel = {\( F / (E A) \), [-]},
        xmin = 0, 
        xmax = 2,
        ymin = -0.15, 
        ymax = 0.2,
        legend cell align=left,
        legend style={anchor=south east, at={(1,0)}}
        ]
        \addplot+ table[skip first n=1] {data/force_deflection_elastic.dat}; % + = same automatically determined styles, but in addition it uses [options]
        \addlegendentry{elastic}
        \addplot+ table[skip first n=1] {data/force_deflection_plastic.dat};
        \addlegendentry{plastic}
      \end{axis}
    \end{tikzpicture}
    \caption{Force - deflection}
  \end{subfigure}

  % Stress - strain:
  \begin{subfigure}[t]{0.48\textwidth}
    \begin{tikzpicture}[baseline,yscale=1.0, xscale=1.0]
      \begin{axis}[
        try min ticks=7,
        minor tick num=1,
        grid=both,
        xlabel = {strain, [-]},
        ylabel = {Kirchhoff stress, [kN/mm\textsuperscript{2}]},
        xmin = -0.4, 
        xmax = 0.3,
        xtick={-0.4, -0.2, 0, 0.2},
        ymin = -30, 
        ymax = 30,
        legend cell align=left,
        legend style={anchor=south east, at={(1,0)}}
        ]
        \addplot+ table[skip first n=1] {data/stress_strain_elastic.dat};
        \addlegendentry{elastic}
        \addplot+ table[skip first n=1] {data/stress_strain_plastic.dat};
        \addlegendentry{plastic}
      \end{axis}
    \end{tikzpicture}
    \caption{Constitutive behaviour}
  \end{subfigure}
  % Total - plastic strain:
  \begin{subfigure}[t]{0.48\textwidth}
    \begin{tikzpicture}[baseline,yscale=1.0, xscale=1.0]
      \begin{axis}[
        % try min ticks=8,
        minor tick num=1,
        grid=both,
        xlabel = {total strain, [-]},
        ylabel = {plastic strain, [-]},
        xmin = -0.4, 
        xmax = 0.3,
        ymin = -0.25, 
        ymax = 0.15
        ]
        \addplot+ [mark=none] table[skip first n=1] {data/tot_pl_strain.dat};
        \addplot+  table[skip first n=1] {data/tot_pl_strain.dat}; % Plot again using style of 2nd element in cycle list.
      \end{axis}
    \end{tikzpicture}  
    \caption{Plastic - total strain}
  \end{subfigure}
  \caption{Plots of large deflection elasto-plastic behaviour from \cite{Bonet2008}.}
  \label{fig:plots-from-book}
\end{figure}


%%% Local Variables:
%%% mode: latex
%%% TeX-master: "../main"
%%% End:
